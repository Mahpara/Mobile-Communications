% header
\documentclass[10pt,a4paper]{article}

\usepackage[utf8]{inputenc}
\usepackage{hyperref}
\usepackage{amssymb}

\usepackage{graphicx}

\usepackage{float}

% the document
\begin{document}

% Ersetzt in den eckigen Klammern bitte die Übungsnummer.
\title{Submission - Exercise [1] [Group\textunderscore15]\\
\large{Mobile Communication}}
\author{ [Mst. Mahfuja Akter , s6msakte@uni-bonn.de] \and [Mahpara Hyder Chowdhury, s6machow@uni-bonn.de]
\and [M Shahzaib, s6mmshah@uni-bonn.de]}
\date{\today}
\maketitle

\section{Free Space and Two-Ray Ground propagation}
\begin{center}
\includegraphics[width=5in,height =3.2in]{images/Capture.PNG}
\end{center}
Answer: We know, by definition
\[ FSP : P_r = \frac{P_t.\lambda^2}{d^2.(4\pi)^2} \]

\[ TRG : P_r = \frac{P_t.h_t^2.h_r^2}{d^4} \]

Therefore,

\[PL_{FSP} = 10log_{10}\frac{P_t}{\frac{P_t.\lambda^2}{d^2.(4\pi)^2}}\newline
= 10log_{10}P_t\frac{d^2.(4\pi)^2}{P_t.\lambda^2}\newline
= 10log_{10}\frac{d^2.(4\pi)^2}{\lambda^2}
\]

\[PL_{TRG}  =  10log_{10}\frac{P_t}{\frac{P_t.h_t^2.h_r^2}{d^4}}\newline
= 10log_{10}P_t\frac{d^4}{P_t.h_t^2.h_r^2}\newline
= 10log_{10}\frac{d^4}{h_t^2.h_r^2}
\]
Hereby, We come to know that path loss for both FSP and TRG are independently specific for $P_t$ and $P_r$ values.
\begin{center}
\includegraphics[width=5in,height =1.2in]{images/Capture2.PNG}
\end{center}

Answer: From the definition of the FSP and TRG for Received power $P_r$ we can write

\[
\frac{P_t.\lambda^2}{d^2.(4\pi)^2} = \frac{P_t.h_t^2.h_r^2}{d^4} \newline
\Rightarrow \frac{P_t.\lambda^2}{P_t.h_t^2.h_r^2} = \frac{d^2.(4\pi)^2}{d^4}\newline
\Rightarrow \frac{\lambda^2}{h_t^2.h_r^2} = \frac{(4\pi)^2}{d^2}\]
\newline
Now considering crossover distance,
\[
\frac{\lambda^2}{h_t^2.h_r^2} = \frac{(4\pi)^2}{\frac{4\pi.h_t.h_r}{\lambda}}
\Rightarrow \frac{\lambda^2}{h_t^2.h_r^2} = \frac{(4\pi.\lamda)}{h_t.h_r}
\Rightarrow \frac{\lambda}{h_t.h_r} = 4\pi
\Rightarrow \frac{4\pi.h_t.h_r}{\lambda} = 0


\]


\section{Comparing two path loss models}
\begin{center}
\includegraphics[width=5in,height =1.7in]{images/Capture3.PNG}
\end{center}
1.  Write a program that implements the path loss calculation using the combined FSP and TRG models with crossover distance dc. In addition, the program should be able to calculate the path loss according to the TLD model.
\newline Answer: \newline
According to our Path loss formula, we made some functions which calculate path loss for specific distance.
\newline Here is our function implementation:
\begin{center}
\includegraphics[width=5.2in,height =3in]{images/PathLoss.PNG}
\end{center}
\begin{center}
\includegraphics[width=5.2in,height =3in]{images/PathLoss2.PNG}
\end{center}
\begin{center}
\includegraphics[width=5.2in,height =1.7in]{images/Capture1.PNG}
\end{center}
Answer: \newline
Using above functions, we measured path loss from 0 meter to 1000 meters and plot those data like below:
\begin{center}
\includegraphics[width=4in,height =2.2in]{images/PathLossPlot.PNG}
\end{center}
We have checked our implemented function with some specific distance also (e.g. 1000,5000,10000)
\newline Here is our measured data:
\begin{center}
\includegraphics[width=4in,height =2.2in]{images/PathLossCheck.PNG}
\end{center}
3.  Give a brief description of your implementation and the resulting plot.\newline
Answer:\newline From our measured data and plot, we can see that the combination of FSP and TRG gives almost same path loss as TLD.
Though FSP works with lower distance and TRG works with high distance, combination of these two gives better result if we can apply crossover distance value properly.
\newline Besides of FSP and TRG, TLD works for all distance and gives similar path loss from other two. So, we can consider TLD  as a complete method.
\section{Reality check}
\begin{center}
\includegraphics[width=5in,height =1in]{images/Capture5.PNG}
\end{center}
Answer: \newline We have converted the geographic coordinates data into latitude, longitude  and distance and pass those distance into path loss function. \newline Here is our implementation:
\begin{center}
\includegraphics[width=5in,height =3in]{images/GeometryCalPL.PNG}
\end{center}
2. Compute the expected path loss for the moving car using both of the models defined before. Produce a plot which shows the expected path loss for both models in comparison to the conducted measurements.
\newline Answer: \newline
We have computed path loss from those distances and normalize those values from 0 to 1. And noticed that FSP/TRG and TLD almost overlaps with each other in normalized form. Besides relative signal works almost inversely with FSP/TRG and TLD. To see the better comparison with conducted relative signal we have inverted this signal and make a plot.\newline Here is our implementation:
\begin{center}
\includegraphics[width=5in,height =1.5in]{images/NormalizeNPlot.PNG}
\end{center}
We found the plot like below:
\begin{center}
\includegraphics[width=5in,height =3in]{images/ComparePL.PNG}
\end{center}
3. Do the path loss models correctly represent the real world signal propagation? Give reasons why this might (not) be the case.
\newline Answer:
\newline No, this model does not represent the real world signal propagation. When relative strength is high, path loss supposed to be low. Here we inverse our relative strength data but till we see that the path loss data is not relative.
\end{document}
